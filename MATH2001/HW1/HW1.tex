\documentclass[a4paper,11pt]{article}
\usepackage{amssymb}
\usepackage{mathtools}
\usepackage{hyperref} 
\usepackage{amsthm}
\usepackage{enumitem}
\usepackage[margin=0.5in]{geometry}
\usepackage[utf8]{inputenc}
\usepackage[linesnumbered,ruled,vlined]{algorithm2e}
\usepackage{listings}
\usepackage{color}
\usepackage[numbers]{natbib}
\usepackage{subfiles}
\usepackage{tkz-berge}


\author{Michelle Tran}
\title{Homework 1}

\begin{document}
\maketitle

\newtheorem{prop}{Proposition}[section]
\newtheorem{thm}{Theorem}[section]
\newtheorem{lemma}{Lemma}[section]
\newtheorem{cor}{Corollary}[prop]

\theoremstyle{definition}
\newtheorem{mydef}{Definition}

\theoremstyle{definition}
\newtheorem{required}{(Recommended) Problem}

\theoremstyle{definition}
\newtheorem{advanced}[required]{(Advanced) Problem}

\theoremstyle{definition}
\newtheorem{challenge}[required]{(Challenge) Problem}

\theoremstyle{definition}
\newtheorem{open}{(Open) Problem}

\theoremstyle{definition}
\newtheorem{ex}{Example}


\section{Proof by Contrapositive} 

\noindent We begin with some definitions.

\begin{mydef}
An integer $n$ is said to be \textit{even} if there is an integer $k$ such that $n = 2k$. Otherwise, $n$ is said to be \textit{odd}.
\end{mydef}

\noindent \textbf{Remark:} It follows that odd integers can be written uniquely as $n = 2k+1$ for some integer $k$. 

\noindent
\begin{mydef}
We say that $x$ is a \textit{rational number}, denoted $x \in \mathbb{Q}$, if there exist integers $p, q$ such that $x = \dfrac{p}{q}$.
\end{mydef}


\noindent 
\begin{required}
Suppose that $a, b \in \mathbb{R}$ such that $ab \not \in \mathbb{Q}$. Then we have $a \not \in \mathbb{Q}$ or $b \not \in \mathbb{Q}$.
\end{required}
% This basically means show that having a product of a and b that doesn't exist in the rationals means that either a or b or both also don't exist in the rationals


\begin{required}
Suppose $x, y \in \mathbb{Z}$ such that $xy$ is even. Then either $x$ is even or $y$ is even.
\end{required}
% As an exercise, try doing this as a direct proof first. Should be able to just make x and y equal to certain even and odd values and trying each case -- but then again, it's a lot more work than doing this by contrapositive. Bc you have three cases to test whereas the contrapositive only has one case.

\begin{required}
Suppose $x, y \in \mathbb{Z}$ such that $x + y \geq 15$. Then either $x \geq 8$ or $y \geq 8$. 
\end{required}
% Yeah this is just a lot easier by contrapositive -- though I'm sure you could try to prove this directly -- it's just longer again

\begin{required}
Suppose $n \in \mathbb{Z}$ such that $n^{3}$ is even. Then $n$ is even.
\end{required}
% Yeah this would be annoying via direct proof

\end{document}