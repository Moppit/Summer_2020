\documentclass[a4paper,11pt]{article}
\usepackage{amssymb}
\usepackage{mathtools}
\usepackage{hyperref} 
\usepackage{amsthm}
\usepackage{enumitem}
\usepackage[margin=0.5in]{geometry}
\usepackage[utf8]{inputenc}
\usepackage[linesnumbered,ruled,vlined]{algorithm2e}
\usepackage{listings}
\usepackage{color}
\usepackage[numbers]{natbib}
\usepackage{subfiles}
\usepackage{tkz-berge}


\author{Michelle Tran}
\title{Homework 1}

\begin{document}
\maketitle

\newtheorem{prop}{Proposition}[section]
\newtheorem{thm}{Theorem}[section]
\newtheorem{lemma}{Lemma}[section]
\newtheorem{cor}{Corollary}[prop]

\theoremstyle{definition}
\newtheorem{mydef}{Definition}

\theoremstyle{definition}
\newtheorem{required}{(Recommended) Problem}

\theoremstyle{definition}
\newtheorem{advanced}[required]{(Advanced) Problem}

\theoremstyle{definition}
\newtheorem{challenge}[required]{(Challenge) Problem}

\theoremstyle{definition}
\newtheorem{open}{(Open) Problem}

\theoremstyle{definition}
\newtheorem{ex}{Example}


\section{Proof by Contrapositive} 

\noindent We begin with some definitions.

\begin{mydef}
An integer $n$ is said to be \textit{even} if there is an integer $k$ such that $n = 2k$. Otherwise, $n$ is said to be \textit{odd}.
\end{mydef}

\noindent \textbf{Remark:} It follows that odd integers can be written uniquely as $n = 2k+1$ for some integer $k$. 

\noindent
\begin{mydef}
We say that $x$ is a \textit{rational number}, denoted $x \in \mathbb{Q}$, if there exist integers $p, q$ such that $x = \dfrac{p}{q}$.
\end{mydef}


\noindent 
\begin{required}
Suppose that $a, b \in \mathbb{R}$ such that $ab \not \in \mathbb{Q}$. Then we have $a \not \in \mathbb{Q}$ or $b \not \in \mathbb{Q}$.
\end{required}

% This basically means show that having a product of a and b that doesn't exist in the rationals means that either a or b or both also don't exist in the rationals
\emph{Proof by Contradiction.} Show that $\neg ((a \not \in \mathbb{Q}) \vee (b \not \in \mathbb{Q})) \Rightarrow \neg (ab \not \in \mathbb{Q})$ where $a,b \in \mathbb{R}.$

$\neg ((a \not \in \mathbb{Q}) \vee (b \not \in \mathbb{Q})) \equiv (a \in \mathbb{Q}) \land (b \in \mathbb{Q})$ by DeMorgan's Law.

$\neg(ab \not \in \mathbb{Q}) \equiv ab \in \mathbb{Q}$ by def. of negation.

Thus, we must show that if a and b are rational, then the product ab is also rational.

Symbolically: $(a \in \mathbb{Q}) \land (b \in \mathbb{Q}) \Rightarrow (ab \in \mathbb{Q})$.

\phantom{}

By Definition 2, a and b can be written as $a = \frac{p}{q}$ and $b = \frac{m}{n}$, where $p,q,m,n \in \mathbb{Z}$.

For $a,b \in \mathbb{Q}$, $ab = \frac{p}{q}*\frac{m}{n} = \frac{pm}{qn}$ where $pm, qn \in \mathbb{Z}$ due to integer multiplication properties.

Therefore, $ab \in \mathbb{Q}$ since $\frac{pm}{qn}$ is a quotient of two integers, as per Definition 2.

As such, we have proven that $(a \in \mathbb{Q}) \land (b \in \mathbb{Q}) \Rightarrow (ab \in \mathbb{Q})$. 

\phantom{}

Hence, by the contrapositive proof, the statement if $ab \not \in \mathbb{Q}$, then we have $a \not \in \mathbb{Q}$ or $b \not \in \mathbb{Q}$, holds. \fbox \\

\begin{required}
Suppose $x, y \in \mathbb{Z}$ such that $xy$ is even. Then either $x$ is even or $y$ is even.
\end{required}
% As an exercise, try doing this as a direct proof first. Should be able to just make x and y equal to certain even and odd values and trying each case -- but then again, it's a lot more work than doing this by contrapositive. Bc you have three cases to test whereas the contrapositive only has one case.
\emph{Proof by Contrapositive.} Show that $\neg$(x is even $\vee$ y is even) $\Rightarrow$ $\neg$(xy is even) where $x, y \in \mathbb{Z}$.

$\equiv$ (x is odd $\land$ y is odd) $\Rightarrow$ (xy is odd) by DeMorgan's Law and the def. of negation.

\phantom{}

By Definition 1, x and y can be written as: $x = 2m + 1, y = 2n + 1$ where $m,n \in \mathbb{Z}$.

$xy = (2m+1)(2n+1) $

$= 4mn + 2m + 2n + 1$

$ = 2(2mn + m + n) + 1 $

$= 2k + 1$ where $k = 2mn + m + n$

Therefore, xy is odd, and as such, we have proven that (x is odd $\land$ y is odd) $\Rightarrow$ (xy is odd).

\phantom{}

Hence, by the contrapositive proof, the statement if $xy$ is even, then either $x$ is even or $y$ is even.   \fbox \\

\begin{required}
Suppose $x, y \in \mathbb{Z}$ such that $x + y \geq 15$. Then either $x \geq 8$ or $y \geq 8$. 
\end{required}
% Yeah this is just a lot easier by contrapositive -- though I'm sure you could try to prove this directly -- it's just longer again
\emph{Proof by Contrapositive.} Show that $\neg((x \geq 8) \vee (y \geq 8)) \Rightarrow \neg((x+y) \geq 15)$ where $x,y \in \mathbb{Z}$.

$\equiv ((x < 8) \land (y < 8)) \Rightarrow ((x+y) < 15)$ by DeMorgan's Law and def. of negation.

\phantom{}

% Not sure how to do this generically? There are infinite values of x and y... all I can think of is using 7 bc it's the highest int on that interval, but it's a specific value... Maybe rewrite this?
Since x and y are integers, we can rewrite the expression $(x < 8) \land (y < 8)$ as $(x \leq 7) \land (y \leq 7)$.

$x+y \leq 7+7 = 14$. Thus, 14 is the greatest possible sum of x and y. % Yeah, this is kind of voodoo... should check this

Since $14 < 15$ is true, we see that it does follow that $(x+y) < 15$.

\phantom{}

Hence, by the contrapositive proof, the statement if $x + y \geq 15$ then either $x \geq 8$ or $y \geq 8$, holds.   \fbox \\

\begin{required}
Suppose $n \in \mathbb{Z}$ such that $n^{3}$ is even. Then $n$ is even.
\end{required}
% Yeah this would be annoying via direct proof
\emph{Proof by Contrapositive.} Show that $\neg$(n is even) $\Rightarrow \neg(n^3$ is even). 

$\equiv$ (n is odd) $\Rightarrow$ ($n^3$ is odd) by the def. of negation.

\phantom{}

Since n is odd, it can be represented generically as $n = 2k+1$ where $k \in \mathbb{Z}$.

Thus, $n^3 = (2k+1)^3$

$ = (4k^2 + 4k + 1)(2k+1) $

$= 8k^3 + 4k^2 + 8k^2 + 4k + 2k + 1$

$= 8k^3 + 12k^2 + 6k + 1$

$= 2(4k^3 + 6k^2 + 3k) + 1$

$= 2t + 1$ where $t = 4k^3 + 6k^2 + 3k$.

Thus, when n is odd, $n^3$ is odd.

Hence, by the contrapositive proof, the statement if $n^3$ is even, then n is even, holds.   \fbox \\

\end{document}